

\documentclass[11pt]{cv_wakeforest_article}\usepackage[]{graphicx}\usepackage[]{color}
% maxwidth is the original width if it is less than linewidth
% otherwise use linewidth (to make sure the graphics do not exceed the margin)
\makeatletter
\def\maxwidth{ %
  \ifdim\Gin@nat@width>\linewidth
    \linewidth
  \else
    \Gin@nat@width
  \fi
}
\makeatother

\definecolor{fgcolor}{rgb}{0.345, 0.345, 0.345}
\newcommand{\hlnum}[1]{\textcolor[rgb]{0.686,0.059,0.569}{#1}}%
\newcommand{\hlstr}[1]{\textcolor[rgb]{0.192,0.494,0.8}{#1}}%
\newcommand{\hlcom}[1]{\textcolor[rgb]{0.678,0.584,0.686}{\textit{#1}}}%
\newcommand{\hlopt}[1]{\textcolor[rgb]{0,0,0}{#1}}%
\newcommand{\hlstd}[1]{\textcolor[rgb]{0.345,0.345,0.345}{#1}}%
\newcommand{\hlkwa}[1]{\textcolor[rgb]{0.161,0.373,0.58}{\textbf{#1}}}%
\newcommand{\hlkwb}[1]{\textcolor[rgb]{0.69,0.353,0.396}{#1}}%
\newcommand{\hlkwc}[1]{\textcolor[rgb]{0.333,0.667,0.333}{#1}}%
\newcommand{\hlkwd}[1]{\textcolor[rgb]{0.737,0.353,0.396}{\textbf{#1}}}%
\let\hlipl\hlkwb

\usepackage{framed}
\makeatletter
\newenvironment{kframe}{%
 \def\at@end@of@kframe{}%
 \ifinner\ifhmode%
  \def\at@end@of@kframe{\end{minipage}}%
  \begin{minipage}{\columnwidth}%
 \fi\fi%
 \def\FrameCommand##1{\hskip\@totalleftmargin \hskip-\fboxsep
 \colorbox{shadecolor}{##1}\hskip-\fboxsep
     % There is no \\@totalrightmargin, so:
     \hskip-\linewidth \hskip-\@totalleftmargin \hskip\columnwidth}%
 \MakeFramed {\advance\hsize-\width
   \@totalleftmargin\z@ \linewidth\hsize
   \@setminipage}}%
 {\par\unskip\endMakeFramed%
 \at@end@of@kframe}
\makeatother

\definecolor{shadecolor}{rgb}{.97, .97, .97}
\definecolor{messagecolor}{rgb}{0, 0, 0}
\definecolor{warningcolor}{rgb}{1, 0, 1}
\definecolor{errorcolor}{rgb}{1, 0, 0}
\newenvironment{knitrout}{}{} % an empty environment to be redefined in TeX

\usepackage{alltt}

\usepackage{hyperref}
\hypersetup{colorlinks=true, linkcolor=blue, 	urlcolor=blue}

\usepackage{cv_wakeforest_style}
\usepackage{amsmath}
\usepackage{setspace}
\usepackage{fancyhdr}
\usepackage{lastpage}
\usepackage{url}
\usepackage{pdfpages}
\usepackage{verbatim}
\usepackage{color}
\usepackage{soul}  %Use \hl{Blah} for notes.
\usepackage{float}

% BCJ Added 11/2/2021 to avoid getting compilation warnings
\setlength{\headheight}{27.32324pt}

% ---- Macros ---- %

\newcommand{\AR}[1]{\hl{#1}}  % highlighting sections for annual review.

% misc formatting
\newcommand{\mysection}[1]{\section*{#1}}
\newcommand{\mysubsection}[1]{\subsection*{#1}}
\newcommand{\myitem}{\item[]}
\newcommand{\supref}[1]{$^\ref{#1}$}

% convenience
\newcommand{\lf}{\\ \> \>}
\newcommand{\me}{{\bfseries Jaeger BC}}
\newcommand{\PMID}[1]{\href{http://www.ncbi.nlm.nih.gov/pubmed/?term=#1}{PMID:  {#1}}}
\newcommand{\PMCID}[1]{\href{http://www.ncbi.nlm.nih.gov/pmc/articles/#1/}{PMCID:  {#1}}}
\newcommand{\DOI}[1]{\href{http://doi.org/#1}{DOI: {#1}}}  %starting in 2010?
\newcommand{\mybox}{\rule{8pt}{8pt}}

\pdfminorversion=4

\rhead{\underline{\large Byron C. Jaeger}\\\today}

% What is this?
\cfoot{\thepage\ of \pageref{LastPageWTA}}

\cfoot{\thepage}

\renewcommand{\headrulewidth}{0pt}

\addtolength{\headheight}{\baselineskip}
\addtolength{\headheight}{1.1pt}
\addtolength{\headsep}{-20pt}

\renewcommand{\sectionmark}[1]{}
\renewcommand{\subsectionmark}[1]{}

\lhead{{\large\bfseries\leftmark}\\{\bfseries\rightmark}}

\setlength{\parindent}{0pt}

\newenvironment{packed_enum}{
\begin{enumerate}
  \setlength{\itemsep}{1pt}
  \setlength{\parskip}{0pt}
  \setlength{\parsep}{0pt}
}{\end{enumerate}}

\newenvironment{packed_item}{
\begin{itemize}
  \setlength{\itemsep}{1pt}
  \setlength{\parskip}{0pt}
  \setlength{\parsep}{0pt}
}{\end{itemize}}
\IfFileExists{upquote.sty}{\usepackage{upquote}}{}
\begin{document}

\pagestyle{fancy}
\thispagestyle{empty}

\begin{center}
{\bfseries\Large 
WAKE FOREST SCHOOL OF MEDICINE \\ CURRICULUM VITAE \\[1em] (\today)
}
\end{center}

\begin{tabbing}

\hspace{1.85 in} \= \hspace{1.05in} \= \\

\bfseries \large NAME

\>\> Byron C. Jaeger, Ph.D. \\

\\

\bfseries \large CURRENT ACADEMIC TITLE

\>\> \AR{Assistant Professor} \\

\\
% {\bfseries\large \AR{ADMINISTRATIVE TITLE}}
% \>\>\AR{Chair, Department of Biostatistics and Data Science}\\
% \\

\bfseries\large  ADDRESS

%\>Residence \> 4101 Greenvale Court\\
%\>\>Winston-Salem, North Carolina 27104\\
%\>\>(336) 760-3410\\
%\\

%\> Business 
\>\>Department of Biostatistics and Data Science\\
\>\>Division of Public Health Sciences\\
\>\>Wake Forest School of Medicine\\
\>\>Medical Center Boulevard\\
\>\>Winston-Salem, North Carolina 27157\\
\>\>(336) 716-6956\\
\>\>{\tt bjaeger@wakehealth.edu}\\
\\


{\bfseries\large  EDUCATION}  \\
\>2008--2012\> Furman University \\
\>\>Greenville, South Carolina \\
\>\>B.S. \emph{summa cum laude} (Mathematics) \\
\\
\>2012--2017\> Gillings School of Global Public Health \\
\>\>Chapel Hill, North Carolina \\
\>\>Ph.D. (Statistics)\\
\\
\>\>Research Advisor:  Lloyd J. Edwards, Ph.D.\\
\>\>Thesis:  \emph{Extending $R^2$ to the generalized linear mixed model}
\end{tabbing}

\mysection{EMPLOYMENT}
\mysubsection{Academic Appointments}
\markboth{EMPLOYMENT (Continued)}{Academic Appointments (Continued)}


\begin{tabbing}

\hskip 0.2in \emph{Gillings School of Global Public Health, Chapel Hill, North Carolina}\\

\hspace{0.25in} \= 2012--2017 \hspace{0.7in} \= Research Assistant, Department of Biostatistics\\

\hspace{0.25in} \= 2014--2017 \hspace{0.7in} \= Teaching Assistant, Department of Biostatistics\\

\hspace{0.25in} \= 2016--2017 \hspace{0.7in} \= Statistical Consultant, School of Nursing\\
\\


\hskip 0.2in \emph{North Carolina Central University, Durham, North Carolina}\\
\>2016--2017\> Adjunct Professor, School of Business\\
\\

\hskip 0.2in \emph{University of Alabama at Birmingham, Birmingham, Alabama}\\
\>2017--2021\> Assistant Professor, Department of Biostatistics\\
\\

\hskip 0.2in \emph{Wake Forest School of Medicine, Wake Forest University, Winston-Salem, North Carolina}\\
\>2021--Present\>Assistant Professor,  Department of Biostatistics and Data Science,
            Division of \\\>\>\hspace{0.25in}Public Health Sciences\\

\end{tabbing}

\mysection{ADMINISTRATIVE SERVICE}

\mysubsection{Institutional  Service}
\begin{tabbing}
\hspace{0.25in} \= 2017--2019\hspace{0.25in} \= Statistical Lecturer, Kirklin Institute of Research in Surgical Outcomes,\\\>\>\hspace{0.25in}University of Alabama at Birmingham (UAB) School of Medicine.\\

\> 2017--2021 \>Academic Advisor for Andrew Sims, UAB Biostatistics \\

\> 2017--2018 \>Member, Masters Committee for Catherine Jones, UAB Epidemiology. ``The association\\\>\>\hspace{0.25in}between physical activity and executive functioning in obese adults'' \\

\> 2018 \>Member, Masters Committee for Nora Ahmad Mohammad Balas, UAB Epidemiolgy.\\\>\>\hspace{0.25in}``Factors associated with breast cancer screening behaviors in a sample of\\\>\>\hspace{0.275in} Jamaican women'' \\

\> 2018--2021 \>Member, Diversity Committee, UAB \\

\> 2018--2020 \>Member, Doctoral Committee for Anastasia Hartzes, UAB Biostatistics.\\\>\>\hspace{0.25in}``Evaluating Lumpability in Markov Models: Applying Markov Processes\\\>\>\hspace{0.275in} to Model Disability in Relapsing Multiple Sclerosis Patients'' \\


\> 2018--2020 \>Member, Doctoral Committee for Steve Ampah, UAB Biostatistics.\\\>\>\hspace{0.25in}``Marginalized Models with Heterogeneous Random Effects for Zero-inflated\\\>\>\hspace{0.275in} Longitudinal Count Data'' \\

\> 2019--2020 \>Member, Undergraduate Honors Thesis Committee for Natalie Simpkins, UAB.\\\>\>\hspace{0.25in}Science and Technology Honors Program \\

\> 2019--2021 \>Member, Doctoral Committee for Charles F. Murchison, UAB Biostatistics.\\\>\>\hspace{0.25in}``Application of Longitudinal Machine Learning Methods in Multi-Study\\\>\>\hspace{0.275in} Alzheimer's Disease Datasets'' \\

\> 2019--2021 \>Member, Doctoral Committee for Boyi Guo, UAB Biostatistics.\\\>\>\hspace{0.25in}``Bayesian Hierarchical Additive Models'' \\


\end{tabbing}

\mysection{EXTRAMURAL APPOINTMENTS AND SERVICE}

\markboth{\ }{\ }

\mysubsection{Editorial Boards}

\begin{enumerate}

\item Member, \textit{Hypertension}, 2019--Present

\item Statistical Section Editor, \textit{Hypertension}, 2022--Present

\end{enumerate}

\mysubsection{Advisory Boards}

\begin{enumerate}

\item Eastern North American Regional Advisory Board Member, 2019--present.

\item Data and Safety Monitoring Board Member, 2021--present, Preventing Cognitive Decline by Reducing Blood Pressure Target Trial, R01 AG068486.

\end{enumerate}

\mysubsection{Journal Reviewer}
\begin{packed_enum}
\item Hypertension
\item Biometrical Journal
\item Biometrics
\item Journal of the American Society of Nephrology
\item Hypertension Research
\item Journal of Human Hypertension
\item Journal of Bone and Mineral Research
\item Circulation: Cardiovascular Quality and Outcomes
\item Mayo Clinic Proceedings
\end{packed_enum}

\mysection{PROFESSIONAL MEMBERSHIPS AND SERVICE}

\mysubsection{Memberships}

\begin{tabbing}
\hspace{0.25in} \= 2011--Present \hspace{0.25in} \=  Member, Eastern North American Region, International Biometric Society\\
\> 2016--Present \> Member, American Statistical Association\\
\end{tabbing}

\mysection{HONORS AND AWARDS}
\begin{tabbing}
\hspace{0.25in} \= 2007\phantom{--Present} \hspace{0.25in} \= Regional finalist, Siemens Competition in Math, Science, and Technology \\
\>2008\> Wylie Scholarship in Mathematics, Furman University\\
\>2012\> Graduated \emph{summa cum laude}, Furman University\\
\>2012\> Phi Beta Kappa, Furman University\\
\>2012\> DeLany Medal for Excellence in Mathematics, Furman University\\
\>2016\> Big Data to Knowledge Scholarship\\
\>2017\> Distinguished Student Paper Award, Eastern North American Region\\
\>2019\> Finalist; American Heart Association Best Artificial Intelligence and Machine\\\>\>\hspace{0.25in}Learning Paper Award 
\end{tabbing}

\mysection{PROFESSIONAL INTERESTS {\normalfont\emph{(Last updated \today)}}}
{\setlength\parindent{1cm}

\hskip 1cm I am an Assistant Professor in the Department of Biostatistics and Data Science in the Division of Public Health Sciences at Wake Forest School of Medicine (WFSM). After receiving my PhD in Statistics at the University of Chapel Hill in North Carolina, I spent four years in the Department of Biostatistics at the University of Alabama at Birmingham School of Public Health and joined the WFSM faculty in 2021. At Birmingham, I was a Biostatistician of the Jackson Heart Study Hypertension Working Group and the American Heart Association Strategically Focused Research Network. I am currently serving as a member of the Regional Advisory Board for the Eastern North American Region of the International Biometric Society.

My current work is focused on computational optimization of the oblique random survival forest, a machine learning algorithm I previously developed. I have studied the etiology of hypertension and cardiovascular disease from the perspective of observational cohort studies that tracked health behaviors, socioeconomic factors, and diet and exercise habits throughout mid-life. TODO: Add details on work at WFSM. 

}

\mysection{GRANTS}
\mysubsection{Current Grants}

%\mysection{GRANTS--CURRENT}
\begin{enumerate}

 \item R01HL117323 (Paul Muntner) \hfill 8/01/2017--8/30/2021
 
 NIH-NHLBI
 
{\bfseries Incorporation of a Hypertension Working Group into the Jackson Heart Study}

The objective of this grant is to identify novel risk factors for hypertension and potential approaches to improve hypertension control rates and reduce BP related complications among African Americans. We will integrate a strong mentorship component to help early stage investigators (ESIs), especially minority investigators, move towards independence.

{\bfseries Role: Biostatistician.}

\item 15SFRN23900002 (Paul Muntner) \hfill 8/01/2017--8/30/2021

American Heart Association

{\bfseries University of Alabama Strategically Focused Hypertension Research Center}

This study focuses on circadian blood pressure patterns. In a population science study, we will evaluate racial differences in nocturnal hypertension and non-dipping blood pressure. In a clinical science project, we will evaluate the effect of sodium intake on diurnal blood pressure and sleep apnea.  In basic science studies, we will study mechanisms leading to loss of diurnal blood pressure patterns.  Additionally, this grant involves the training of three post-doctoral fellows.

{\bfseries Role: Biostatistician.}


\end{enumerate}

\mysection{BIBLIOGRAPHY}




My MyNCBI collection is available publicly  \href{https://www.ncbi.nlm.nih.gov/myncbi/byron.jaeger.1/bibliography/public/}{here}.
As of \today, my h-index was 14 with 1073 total citations. More details can be accessed from my \href{https://scholar.google.com/citations?user=4IKD_roAAAAJ&hl=en}{scholar profile}.

\mysubsection{Peer-Reviewed Journal Articles}

\begin{enumerate}

\item \me, Lewis TM. Enumerating graph depletions. Online Journal of Analytic Combinatorics. 2014;9. 

\item Dampier C, \me, Gross HE, Barry V, Edwards L, Lui Y, DeWalt DA, Reeve BB. Responsiveness of PROMIS Pediatric Measures to Hospitalizations for Sickle Pain and Subsequent Recovery. Pediatr Blood Cancer. 2016 Jun;63(6):1038-45. \DOI{10.1002/pbc.25931}. Epub 2016 Feb 8. \PMID{26853841}; \PMCID{PMC5055833}.

\item Reeve BB, Edwards LJ, \me, Hinds PS, Dampier C, Gipson DS, Selewski DT, Troost JP, Thissen D, Barry V, Gross HE, DeWalt DA. Assessing responsiveness over time of the PROMIS® pediatric symptom and function measures in cancer, nephrotic syndrome, and sickle cell disease. Qual Life Res. 2018 Jan;27(1):249-257. \DOI{10.1007/s11136-017-1697-z}. Epub 2017 Sep 7. \PMID{28884421}; \PMCID{PMC5771815}.

\item \me, Edwards LJ, Das K, Sen PK. An $R^2$ statistic for fixed effects in the generalized linear mixed model, Journal of Applied Statistics. Epub 2016 June; 44(6):1086-1105. \DOI{10.1080/02664763.2016.1193725}.

\item Smith AB, \me, Pinheiro LC, Edwards LJ, Tan HJ, Nielsen ME, Reeve BB. Impact of bladder cancer on health-related quality of life. BJU Int. 2018 Apr;121(4):549-557. \DOI{10.1111/bju.14047}. Epub 2017 Nov 1. \PMID{28990272}.

% abstracts
% 12. Narang GL, Smith AB, Jaeger B, Pinheiro L, Edwards LJ, Pruthi R, et al. Changes in Health-Related Quality of Life Among Older Adults with Prostate Cancer: A Case-Control Study. Journal of the American College of Surgeons. 2017;225(4):e105. 
% 13. Smith A, Jaeger B, Pinheiro L, Edwards L, Reeve B. MP14-13 CHANGES IN HEALTH-RELATED QUALITY OF LIFE AMONG OLDER ADULTS WITH PROSTATE CANCER: A CASE-CONTROL STUDY. The Journal of Urology. 2017;197(4S):e166–7. 
% 21. Butler M, Kalinowski J, Shimbo D, Sims M, Booth JN, Bress AP, et al. Abstract P179: Hypertension Awareness is Associated With Negative Psychosocial Outcomes in Africans Americans in the Jackson Heart Study (JHS). Circulation. 2019;139(Suppl_1):AP179–AP179. 
% 30. Roberts LM, Jaeger BC, Baptista LC, Harper SA, Jackson E, Gardner AK, et al. Wearable Technology To Reduce Sedentary Behavior And CVD Risk In Older Adults: A Pilot Trial: 3620 Board# 308 June 1 9: 30 AM-11: 00 AM. Medicine & Science in Sports & Exercise. 2019;51(6S):1004. 
% \item Siddiqui M, \me, Gaddam K, Dudenbostel T, Judd E, Pollock D, et al. Spironolactone does not change 24 hour blood pressure circadian pattern in resistant hypertensive patients. Journal of Hypertension. 2019;37:e313. 
% \item Siddiqui M, Jaeger BC, Judd E, Dudenbostel T, Pollock DM, Pollock J, et al. Novel 24-hour blood pressure profile in patients with resistant hypertension. Journal of the American College of Cardiology. 2019;73(9S1):1818–1818.
% 42. Jaeger BC, Cantor RS, Sthanam V, Rudraraju R. Improving Mortality Predictions for Patients With Mechanical Circulatory Support Using Follow-Up Data and Machine Learning. Circulation: Genomic and Precision Medicine. 2020;13(4):e002877. 


% tutorials
% 40. Jaeger BC. The Use of Statistical Learning in Pediatric Research. The Journal of Pediatrics. 2020;223:231–3. 


\item Webster-Clark M, \me, Zhong Y, Filler G, Alvarez-Elias A, Franceschini N, Díaz-González de Ferris ME. Low agreement between modified-Schwartz and CKD-EPI eGFR in young adults: a retrospective longitudinal cohort study. BMC Nephrol. 2018 Aug 6;19(1):194. \DOI{10.1186/s12882-018-0995-1}. \PMID{30081844}; \PMCID{PMC6080537.}

\item Baptista LC, \me, Anton SD, Bavry AA, Handberg EM, Gardner AK, Harper SA, Roberts LM, Sandesara B, Carter CS, Buford TW. Multimodal Intervention to Improve Functional Status in Hypertensive Older Adults: A Pilot Randomized Controlled Trial. J Clin Med. 2019 Feb 6;8(2):196. \DOI{10.3390/jcm8020196}. \PMID{30736317}; \PMCID{PMC6406861}.

\item Bittner V, Colantonio LD, Dai Y, Woodward M, Mefford MT, Rosenson RS, Muntner P, Monda KL, Kilgore ML, \me, Levitan EB. Association of Region and Hospital and Patient Characteristics With Use of High-Intensity Statins After Myocardial Infarction Among Medicare Beneficiaries. JAMA Cardiol. 2019 Sep 1;4(9):865-872. \DOI{10.1001/jamacardio.2019.2481}. \PMID{31339519}; \PMCID{PMC6659160}.

\item Booth JN, Anstey DE, Bello NA, \me, Pugliese DN, Thomas SJ, Deng L, Shikany JM, Lloyd-Jones D, Schwartz JE, Lewis CE, Shimbo D, Muntner P. Race and sex differences in asleep blood pressure: The Coronary Artery Risk Development in Young Adults (CARDIA) study. J Clin Hypertens (Greenwich). 2019 Feb;21(2):184-192. \DOI{10.1111/jch.13474}. Epub 2019 Feb 5. \PMID{30719843}; \PMCID{PMC6375074}.

\item Bowling CB, Deng L, Sakhuja S, Morey MC, \me, Muntner P. Prevalence of Activity Limitations and Association with Multimorbidity Among US Adults 50 to 64 Years Old. J Gen Intern Med. 2019 Nov;34(11):2390-2396. DOI{10.1007/s11606-019-05244-8}. Epub 2019 Aug 21. \PMID{31435766}; \PMCID{PMC6848639}.

\item Harper SA, Roberts LM, Layne AS, \me, Gardner AK, Sibille KT, Wu SS, Vincent KR, Fillingim RB, Manini TM, Buford TW. Blood-Flow Restriction Resistance Exercise for Older Adults with Knee Osteoarthritis: A Pilot Randomized Clinical Trial. J Clin Med. 2019 Feb 21;8(2):265. \DOI{10.3390/jcm8020265}. \PMID{30795545}; \PMCID{PMC6406824}.

\item Hubbard D, Colantonio LD, Tanner RM, Carson AP, Sakhuja S, \me, Carey RM, Cohen LP, Shimbo D, Butler M, Bertoni AG, Langford AT, Booth JN 3rd, Kalinowski J, Muntner P. Prediabetes and Risk for Cardiovascular Disease by Hypertension Status in Black Adults: The Jackson Heart Study. Diabetes Care. 2019 Dec;42(12):2322-2329. \DOI{10.2337/dc19-1074}. Epub 2019 Oct 7. \PMID{31591089}; \PMCID{PMC7011196}.

\item \me, Anstey DE, Bress AP, Booth JN 3rd, Butler M, Clark D 3rd, Howard G, Kalinowski J, Long DL, Ogedegbe G, Plante TB, Shimbo D, Sims M, Supiano MA, Whelton PK, Muntner P. Cardiovascular Disease and Mortality in Adults Aged $\geq$ 60 Years According to Recommendations by the American College of Cardiology/American Heart Association and American College of Physicians/American Academy of Family Physicians. Hypertension. 2019 Feb;73(2):327-334. \DOI{10.1161/HYPERTENSIONAHA.118.12291}. \PMID{30595115}; \PMCID{PMC6392064}. 

\item \me, Edwards LJ, Gurka MJ. An $R^2$ statistic for covariance model selection in the linear mixed model. Journal of Applied Statistics. 2019;46(1):164–84. \DOI{10.1080/02664763.2018.1466869}.

\item \me, Long DL, Long DM, Sims M, Szychowski JM, Min Y-I, et al. Oblique random survival forests. The Annals of Applied Statistics. 2019;13(3):1847–83. \DOI{10.1214/19-AOAS1261}.

\item Pugliese DN, Booth JN 3rd, Deng L, Anstey DE, Bello NA, \me, Shikany JM, Lloyd-Jones D, Lewis CE, Schwartz JE, Muntner P, Shimbo D. Sex differences in masked hypertension: the Coronary Artery Risk Development in Young Adults study. J Hypertens. 2019 Dec;37(12):2380-2388. \DOI{10.1097/HJH.0000000000002175}. \PMID{31246891}; \PMCID{PMC7006727}.

\item Roberts LM, \me, Baptista LC, Harper SA, Gardner AK, Jackson EA, Pekmezi D, Sandesara B, Manini TM, Anton SD, Buford TW. Wearable Technology To Reduce Sedentary Behavior And CVD Risk In Older Adults: A Pilot Randomized Clinical Trial. Clin Interv Aging. 2019 Oct 23;14:1817-1828. \DOI{10.2147/CIA.S222655}. \PMID{31695350}; \PMCID{PMC6815758}.

\item Siddiqui M, Judd EK, \me, Bhatt H, Dudenbostel T, Zhang B, Edwards LJ, Oparil S, Calhoun DA. Out-of-Clinic Sympathetic Activity Is Increased in Patients With Masked Uncontrolled Hypertension. Hypertension. 2019 Jan;73(1):132-141. \DOI{10.1161/HYPERTENSIONAHA.118.11818}. \PMID{30571547}; \PMCID{PMC6309788}.

\item Yano Y, Tanner RM, Sakhuja S, \me, Booth JN 3rd, Abdalla M, Pugliese D, Seals SR, Ogedegbe G, Jones DW, Muntner P, Shimbo D. Association of Daytime and Nighttime Blood Pressure With Cardiovascular Disease Events Among African American Individuals. JAMA Cardiol. 2019 Sep 1;4(9):910-917. \DOI{10.1001/jamacardio.2019.2845}. Erratum in: JAMA Cardiol. 2019 Sep 1;4(9):953. \PMID{31411629}; \PMCID{PMC6694386}.

\item Akinyelure OP, Sakhuja S, Colvin CL, Clark D 3rd, \me, Hardy ST, Howard G, Cohen LP, Irvin MR, Tanner R, Carey RM, Muntner P. Cardiovascular Health and Transition From Controlled Blood Pressure to Apparent Treatment Resistant Hypertension: The Jackson Heart Study and the REGARDS Study. Hypertension. 2020 Dec;76(6):1953-1961. \DOI{10.1161/HYPERTENSIONAHA.120.15890}. Epub 2020 Nov 2. \PMID{33131312}.

% \item Balas N, Yun H, \me, Aung M, Jolly PE. Factors associated with breast cancer screening behaviors in a sample of Jamaican women in 2013. Women & health. 2020;60(9):1032–9. 
% 
% \item Bello NA, \me, Booth III JN, Abdalla M, Anstey DE, Pugliese DN, et al. Associations of Awake and Asleep Blood Pressure and Blood Pressure Dipping with Abnormalities of Cardiac Structure: The Coronary Artery Risk Development in Young Adults (CARDIA) Study. Journal of hypertension. 2020;38(1):102. 
% 
% \item Bodduluri S, Nakhmani A, Reinhardt JM, Wilson CG, McDonald M-L, Rudraraju R, \me, et al. Deep neural network analyses of spirometry for structural phenotyping of chronic obstructive pulmonary disease. JCI insight. 2020;5(13). 
% 
% \item Booth III JN, \me, Huang L, Abdalla M, Sims M, Butler M, et al. Morning blood pressure surge and cardiovascular disease events and all-cause mortality in blacks: The Jackson Heart Study. Hypertension. 2020;75(3):835–43. 
% 
% \item \me, Booth III JN, Butler M, Edwards LJ, Lewis CE, Lloyd-Jones DM, et al. Development of predictive equations for nocturnal hypertension and nondipping systolic blood pressure. Journal of the American Heart Association. 2020;9(2):e013696. 


% 43. Jaeger BC, Tierney NJ, Simon NR. When to Impute? Imputation before and during cross-validation. arXiv preprint arXiv:201000718. 2020; 

% 44. Langford AT, Akinyelure OP, Moore Jr TL, Howard G, Min Y-I, Hillegass WB, et al. Underutilization of treatment for black adults with apparent treatment-resistant hypertension: JHS and the REGARDS study. Hypertension. 2020;76(5):1600–7. 
% 45. Muntner P, Hardy ST, Fine LJ, Jaeger BC, Wozniak G, Levitan EB, et al. Trends in blood pressure control among US adults with hypertension, 1999-2000 to 2017-2018. Jama. 2020;324(12):1190–200. 
% 46. Sakhuja S, Booth JN, Anstey DE, Jaeger BC, Lewis CE, Lloyd-Jones DM, et al. Using predicted atherosclerotic cardiovascular disease risk for discrimination of awake or nocturnal hypertension. American Journal of Hypertension. 2020;33(11):1011–20. 
% 47. Sakhuja S, Hardy ST, Akinyelure P, Jaeger BC, Shimbo D, Bress AP, et al. Abstract P335: Using Cardiovascular Disease Risk Versus Blood Pressure To Guide The Initiation Of Antihypertensive Medication Among Us Adults. Circulation. 2020;141(Suppl_1):AP335–AP335. 
% 48. Sakhuja S, Hardy ST, Akinyelure P, Jaeger BC, Shimbo D, Bress AP, et al. Using Cardiovascular Disease Risk Versus Blood Pressure To Guide The Initiation Of Antihypertensive Medication Among Us Adults. In: CIRCULATION. LIPPINCOTT WILLIAMS & WILKINS TWO COMMERCE SQ, 2001 MARKET ST, PHILADELPHIA …; 2020. 
% 49. Segar MW, Jaeger B, Patel KV, Nambi V, Ndumele CE, Correa A, et al. Development and Validation of Machine Learning-based Race-specific Models to Predict 10-year Risk of Heart Failure: A Multi-cohort Analysis. Circulation. 2020;142(Suppl_3):A196–A196. 
% 50. Thomas SJ, Booth III JN, Jaeger BC, Hubbard D, Sakhuja S, Abdalla M, et al. Association of sleep characteristics with nocturnal hypertension and nondipping blood pressure in the CARDIA study. Journal of the American Heart Association. 2020;9(7):e015062. 
% 51. Wisotzkey BL, Asante-Korang A, Kirklin JK, Cantor R, Koehl D, Jaeger B, et al. Risk Factors for One-year Mortality and Allograft Loss in Pediatric Heart Transplant Patients Using Machine Learning: Analysis of the Pediatric Heart Transplant Society Database. Circulation. 2020;142(Suppl_3):A14239–A14239. 
% 52. Yano Y, Poudel B, Chen L, Sakhuja S, Jaeger B, Viera A, et al. Abstract P173: The Impact of Asleep Blood Pressure on the Prevalence of Masked Hypertension by Race/ethnicity: Analysis of Pooled Population-and Community-based Studies. Circulation. 2020;141(Suppl_1):AP173–AP173. 
% 53. Yano Y, Poudel B, Chen L, Sakhuja S, Jaeger B, Viera A, et al. The Impact of Asleep Blood Pressure on the Prevalence of Masked Hypertension by Race/ethnicity: Analysis of Pooled Population-and Community-based Studies. In: CIRCULATION. LIPPINCOTT WILLIAMS & WILKINS TWO COMMERCE SQ, 2001 MARKET ST, PHILADELPHIA …; 2020. 
% 54. Akinyelure OP, Jaeger BC, Moore TL, Hubbard D, Oparil S, Howard VJ, et al. Racial Differences in Blood Pressure Control Following Stroke: The REGARDS Study. Stroke. 2021;52(12):3944–52. 
% 55. Awad MZ, Vaden RJ, Irwin ZT, Gonzalez CL, Black S, Nakhmani A, et al. Subcortical short-term plasticity elicited by deep brain stimulation. Annals of clinical and translational neurology. 2021;8(5):1010–23. 
% 56. Bundy JD, Jaeger BC, Huffman MD, Knox SS, Thomas SJ, Shimbo D, et al. Twenty-Five-Year Changes in Office and Ambulatory Blood Pressure: Results From the Coronary Artery Risk Development in Young Adults (CARDIA) Study. American Journal of Hypertension. 2021;34(5):494–503. 
% 57. Butts RJ, Kirklin J, Cantor R, Zhao H, Jaeger B, Nandi D, et al. Congenital Heart Disease Classification Improves SRTR Risk Estimation of Waitlist and Early Post-Transplant Mortality for Pediatric Heart Transplantation. Circulation. 2021;144(Suppl_1):A11567–A11567. 
% 58. Cohen LP, Hubbard D, Colvin CL, Jaeger BC, Poudel B, Abdalla M, et al. Lifestyle behaviors among adults recommended for ambulatory blood pressure monitoring according to the 2017 ACC/AHA blood pressure guideline. American Journal of Hypertension. 2021;34(11):1181–8. 
% 59. Goyal P, Yum B, Navid P, Chen L, Kim DH, Roh J, et al. Frailty and Post-hospitalization Outcomes in Patients With Heart Failure With Preserved Ejection Fraction. The American Journal of Cardiology. 2021;148:84–93. 
% 60. Guo B, Jaeger BC, Rahman A, Long DL, Yi N. Spike-and-Slab Generalized Additive Models and Scalable Algorithms for High-Dimensional Data. arXiv preprint arXiv:211014449. 2021; 
% 61. Hardy ST, Chen L, Cherrington AL, Moise N, Jaeger BC, Foti K, et al. Racial and Ethnic Differences in Blood Pressure Among US Adults, 1999–2018. Hypertension. 2021;78(6):1730–41. 
% 62. Hardy ST, Sakhuja S, Jaeger BC, Oparil S, Akinyelure OP, Spruill TM, et al. Maintaining Normal Blood Pressure Across the Life Course: The JHS. Hypertension. 2021;77(5):1490–9. 
% 63. Hardy ST, Sakhuja S, Jaeger BC, Urbina EM, Suglia SF, Feig DI, et al. Trends in Blood Pressure and Hypertension Among US Children and Adolescents, 1999-2018. JAMA network open. 2021;4(4):e213917–e213917. 
% 64. Harper SA, Bassler JR, Peramsetty S, Yang Y, Roberts LM, Drummer D, et al. Resveratrol and exercise combined to treat functional limitations in late life: A pilot randomized controlled trial. Experimental gerontology. 2021;143:111111. 
% 65. Jaeger BC, Akinyelure OP, Sakhuja S, Bundy JD, Lewis CE, Yano Y, et al. Number and timing of ambulatory blood pressure monitoring measurements. Hypertension Research. 2021;44(12):1578–88. 
% 66. Jaeger BC, Cantor R, Sthanam V, Xie R, Kirklin JK, Rudraraju R. Improving outcome predictions for patients receiving mechanical circulatory support by optimizing imputation of missing values. Circulation: Cardiovascular Quality and Outcomes. 2021;14(9):e007071. 
% 67. Li S, Schwartz JE, Shimbo D, Muntner P, Shikany JM, Booth JN, et al. Estimated Prevalence of Masked Asleep Hypertension in US Adults. JAMA cardiology. 2021;6(5):568–73. 
% 68. Muntner P, Jaeger BC, Hardy ST, Foti K, Reynolds K, Whelton PK, et al. Age-specific prevalence and factors associated with normal blood pressure among US adults. American journal of hypertension. 2021; 
% 69. Sakhuja S, Colvin CL, Akinyelure OP, Jaeger BC, Foti K, Oparil S, et al. Reasons for Uncontrolled Blood Pressure Among US Adults: Data From the US National Health and Nutrition Examination Survey. Hypertension. 2021;78(5):1567–76. 
% 70. Segar MW, Jaeger BC, Patel KV, Nambi V, Ndumele CE, Correa A, et al. Development and Validation of Machine Learning-Based Race-Specific Models to Predict 10-Year Risk of Heart Failure: A Multi-Cohort Analysis. Circulation. 2021; 
% 71. Segar MW, Patel KV, Vaduganathan M, Caughey MC, Jaeger BC, Basit M, et al. Development and validation of optimal phenomapping methods to estimate long-term atherosclerotic cardiovascular disease risk in patients with type 2 diabetes. Diabetologia. 2021;64(7):1583–94. 
% 72. Tajeu GS, Colvin C, Hardy ST, Gaye B, Bress AP, Jaeger B, et al. Prevalence, Risk Factors And Cardiovascular Outcomes Associated With Persistent Blood Pressure Control: The Jackson Heart Study. Circulation. 2021;143(Suppl_1):A064–A064. 
% 73. Turkson-Ocran R-AN, Miller ER, Zhao D, Pilla S, Duan D, Jaeger B, et al. Abstract MP14: The Effect Of Time-restricted Feeding On 24-hour Ambulatory Blood Pressure: Results From The Time-restricted Intake Of Meals (TRIM) Study. Circulation. 2021;143(Suppl_1):AMP14–AMP14. 
% 74. Turkson-Ocran R-AN, Miller ER, Zhao D, Pilla S, Duan D, Jaeger B, et al. The Effect Of Time-restricted Feeding On 24-hour Ambulatory Blood Pressure: Results From The Time-restricted Intake Of Meals (TRIM) Study. In: CIRCULATION. LIPPINCOTT WILLIAMS & WILKINS TWO COMMERCE SQ, 2001 MARKET ST, PHILADELPHIA …; 2021. 
% 75. Zhang Y, Schwartz JE, Jaeger BC, An J, Bellows BK, Clark III D, et al. Association Between Ambulatory Blood Pressure and Coronary Artery Calcification: The JHS. Hypertension. 2021;77(6):1886–94. 
% 76. Jaeger BC, Sakhuja S, Hardy ST, Akinyelure OP, Bundy JD, Muntner P, et al. Predicted cardiovascular risk for United States adults with diabetes, chronic kidney disease, and at least 65 years of age. Journal of Hypertension. 2022;40(1):94–101. 
% 77. Kim J, Arora P, Kwon SY, Parcha V, Levitan EB, Jaeger BC, et al. Relation of Abdominal Obesity to Risk of Atrial Fibrillation (From the Reasons for Geographic and Racial Differences in Stroke [REGARDS] Study). The American journal of cardiology. 2022;162:116–21. 
% 78. Long DL, Guo B, McClure LA, Jaeger BC, Tison SE, Howard G, et al. Biomarkers as MEDiators of racial disparities in risk factors (BioMedioR): Rationale, study design, and statistical considerations. Annals of epidemiology. 2022;66:13–9. 
% 79. Bhatt SP, Balte PP, Schwartz JE, Jaeger BC, Cassano PA, Chaves PH, et al. Pooled Cohort Probability Score for Subclinical Airflow Obstruction. 


\end{enumerate}

\end{document}
